\documentclass[a4paper]{article}

\usepackage[spanish]{babel}
\usepackage{graphicx}
\usepackage{amsmath, amssymb}
\usepackage[margin=2.5cm]{geometry}
\usepackage{fancyhdr}
\usepackage{enumerate}
\usepackage[shortlabels]{enumitem}
\usepackage{parskip}
\usepackage[most]{tcolorbox}
\usepackage[hidelinks]{hyperref}
\usepackage{float}
\usepackage{xcolor}
\usepackage{afterpage}
\usepackage{setspace}

% cabecera
\pagestyle{fancy}
\fancyhead[l]{Semillero de Astronomía}
\fancyhead[c]{Banco de Problemas \#1}
\fancyhead[r]{\today}
\fancyfoot[c]{\thepage}
\renewcommand{\headrulewidth}{0.2pt}

% Comando para espacio vertical
\newcommand{\vsp}{\vspace{0.5cm}}

\begin{document}

% Portada
\begin{titlepage}
    \centering
    
    \vspace*{2cm}
    
    % Título principal - color negro normal
    {\fontsize{36}{40}\selectfont\bfseries BANCO DE PROBLEMAS}
    
    \vsp
    \vsp
    
    % Subtítulo
    {\Large\textbf{Astronomía para las Ciencias Básicas}}
    
    \vsp
    
    {\large Volumen 1}
    
    \vspace{3cm}
    
    \rule{0.7\textwidth}{0.4pt}
    
    \vspace{2.5cm}
    
    {\LARGE\bfseries Institución Educativa}
    
    \vsp
    
    {\LARGE\bfseries Enrique Olaya Herrera}
    
    \vsp
    
    {\LARGE\bfseries (IEEOH)}
    
    \vspace{2.5cm}
    
    {\Large Una iniciativa del:}
    
    \vspace{0.8cm}
    
    \begin{minipage}{0.8\textwidth}
        \centering
        {\fontsize{20}{24}\selectfont\bfseries
        CENTRO DE INTERÉS:}
        
        \vspace{0.4cm}
        
        {\fontsize{24}{28}\selectfont\bfseries
        SEMILLERO DE ASTRONOMÍA}
    \end{minipage}
    
    \vfill
    
    \begin{minipage}{\textwidth}
        \centering
        {\large \today}
    \end{minipage}
    
    \nopagecolor
    
\end{titlepage}

\section{Fuerza de Marea}

    \begin{figure}[H]
        \centering
        \includegraphics[width=0.7\textwidth]{TidalForce.png}
        \caption{Fuerza de marea entre la tierra y la luna}
        \label{fig:1}
    \end{figure}

  Una fuerza de marea es una diferencia en la intensidad de la gravedad entre dos puntos.
    El campo gravitacional de la luna produce una fuerza de marea a lo largo del diámetro de la Tierra, 
    lo que causa que la Tierra se deforme. También generan mareas de varios metros en la Tierra sólida, 
    y mareas aún más grandes en los océanos líquidos. La Figura \ref{fig:1} muestra una idea de la situación.

    Un ser humano cayendo en un agujero negro también experimentará fuerzas de marea. 
    ¡En la mayoría de los casos estas serán letales! La diferencia en la fuerza gravitacional 
    entre la cabeza y los pies podría ser tan intensa que una persona literalmente sería separada por tracción. 
    Algunos físicos han llamado a este proceso ¡\textit{espaguetificación}! 

    
    \begin{equation}
    a = \frac{2GMd}{R^3}
    \label{eq:1}
    \end{equation}

    \begin{enumerate}
        \item La ecuación \eqref{eq:1} nos permite calcular la aceleración de marea $a$, através de un
        cuerpo de longitud $d$. La aceleración de marea entre tu cabeza y tus pies está dada por la fórmula \ref{eq:1}. 
        Para $M = 5.9\times 10^{27} \text{ g}$ (masa de la tierra), $R=6.4\times 10^8 \text{cm}$ (radio de la tierra) y
        $G=6.67\times 10^{-8} \text{ dyn} \cdot \text{cm}^2/\text{g}^2$, calcula la aceleración de marea, $a$, 
        si una altura humana típica es $d=200\text{ cm}$.
        \item ¿Cuál es la aceleración de marea a través del diámetro completo de la Tierra?
        \item Un agujero negro de masa estelar tiene la masa del sol $(1.9\times 10^{33}\text{ g)}$, 
        y un radio de $2.9 \text{km}$.
        \begin{enumerate}
            \item A una distancia de $100km$, ¿cuál sería la aceleración de marea através de un humano para $d=200cm$
            \item Si la aceleración de gravedad en la superficie de la tierra es $980 	\text{cm/s}^2$, 
            ¿sería espaguetificado el desafortunado viajero humano cerca de un agujero negro de masa estelar?
        \end{enumerate}
        \item Un agujero negro supermasivo tiene $100$ millones de veces la masa del sol, y un radio de $295$ millones de kilómetros.
        ¿Cuál sería la aceleración de marea a través de un humano con $d=2\text{m}$, a una distancia de $100\text{km}$ desde el horizonte 
        de eventos del agujero negro supermasivo?
        \item En qué agujero negro podría entrar un humano sin ser espaguetificado?
    \end{enumerate}

\newpage

\section{Constantes Físicas}

    Aunque solo existen una docena de constantes físicas fundamentales de la naturalez,
    estas se pueden combinar para definir muchas otras constantes básicas en física, 
    química y astronomía.

    \begin{table}[H]
    \centering
    \begin{tabular}{|c|c|c|}
    \hline
    Símbolo & Nombre & Valor \\
    \hline
    $c$ & Velocidad de la Luz & $2.9979\times 10^{10} \text{ cm/s}$\\
    \hline
    $h$ & Constante de Planck & $6.6262\times 10^{-27} \text{ erg}\cdot \text{s}$\\
    \hline
    $m$ & Masa del Electrón & $9.1095\times 10^{-28} \text{ g}$\\
    \hline
    $e$ & Carga del Electrón & $4.80325 \times 10^{-10} \text{esu}$\\
    \hline
    $G$ & Constante de Gravitación & $6.6732 \times 10^{-8} \text{dyn} \cdot \text{cm}^2 \text{gm}^{-2}$\\
    \hline
    $M$ & Masa del Protón & $1.6726 \times 10^{-24} \text{g}$\\
    \hline
    \end{tabular}
    \caption{Constantes físicas}
    \label{tab:1}
    \end{table}
    En este ejercicio, calcularás algunas de estas constantes \textit{secundarias} con una precisión 
    de tres cifras significativas utilizando una calculadora o programación y los valores definidos 
    en la tabla \ref{tab:1}.

    \begin{enumerate}
        \item Constante de entropía de agujero negro $$\frac{c^3}{2hG}$$
        \item Constante de radiación gravitacional $$\frac{32 G^5}{5 c^{10}}$$
        \item Constante de Thomas-Fermi $$\frac{324}{175} \left(\frac{4}{9\pi}\right)^{2/3}$$
        \item Sección transversal de dispersión de Thompson $$\frac{8 \pi}{3} \left(\frac{e^2}{mc^2}\right)^2$$
        \item Límite de Stark $$\frac{1}{M^5} \left(\frac{4 \pi^2 e^2 m}{h^2}\right)^2$$
        \item Constante de radiación de Bremstrahlung $$\frac{32 \pi^2 e^6}{3\sqrt{2\pi}m^3c}$$
        \item Constante de Fotoionización $$\frac{32 \pi^2 e^6 (2\pi^2 e^4 m)}{3^{3/2} h^3}$$
    \end{enumerate}
    
\newpage

\section{Masa Lunar}

El $19$ de Julio de $1969$, el Módulo de Servicio y Comando Apollo-11 y el Módulo Lunar Eagle entraron en órbita lunar.
    \begin{figure}[H]
    \centering
    \includegraphics[width=0.5\textwidth]{moon.png}
    \label{fig:2}
    \end{figure}
    El tiempo necesario para completar una vuelta completa en la órbita se llama período orbital, 
    que en este caso fue de $2$ horas, a una distancia de $1.737$ km desde el centro de la Luna.

    ¡Crea o no lo creas, puedes usar estas dos piezas de información para determinar la masa de la Luna!
    Así es como se hace:

    \begin{enumerate}
        \item Suponga que el Apollo-11 entró en una órbita circular, y que la aceleración gravitacional hacia
        adentro ejercida por la Luna sobre la cápsula, $F_g/m$, equilibra exactamente la aeleración centrífuga
        hacia afuera, $F_c/m$. Resuelva $F_c=F_g$ para encontrar la masa de la Luna, $M$, en términos de $V$, $R$
        y la constante gravitacional $G$, dado que:
        $$F_g = \frac{GMm}{R^2}\quad\quad F_c = \frac{mV^2}{R}$$
        \item Usando el hecho que para el movimiento circular, $$V = \frac{2\pi R}{T}$$ reexprese su respuesta
        al problema 1 en términos de $R,T$ y $M$.
        \item Dado que $G=6.67\times 10^{-11} \text{ m}^3 \text{kg}^{-1} \text{s}^{-2}$, $R=1.737 \text{ km}$ 
        y $T=2\text{ h}$, calcule la masa de la Luna, $M$, en kilogramos.
        \item La masa de la Tierra es $M = 5.97\times 10^{24} \text{kg}$. ¿Cuál es la razón de la masa de la luna,
        derivada del problema 3, respecto a la pasa de la tierra?
    \end{enumerate}
    
    
\newpage
\section{Temperatura de Equilibrio}

A medida que un cuerpo absorbe energía que incide
    sobre su superficie, también emite enería de vuelta al espacio. 
    Cuando la \textit{energía entrante} iguala a la \textit{energía saliente}, 
    el cuerpo mantiene una temperatura constante de \textit{equilibrio} 

    Si el cuerpo absorbe el $100\%$ de la energía que incide sobre él, 
    la relación entre energía absorbida en $\text{W/m}^2$, $F$, 
    y la temperatura de equilibrio medida en $\text{K}$, $T$, 
    está dada por $$F = 5.7\times 10^{-8} T^4$$
   
    \begin{figure}[H]
        \centering
        \includegraphics[width=0.7\textwidth]{EnceladusTemperatureMap.jpg}
        \caption{Este mapa de temperaturas del satélite Encélado fue creado a partir de los
        datos infrarrojos de la nave espacial Cassini de la NASA}
        \label{fig:1}
    \end{figure}

    \begin{enumerate}
        \item Un cuerpo humano tiene un área superficial de $2\text{ m}^2$ y se encuentra a una 
        temperatura de $98.6\text{ °F}$. 
        ¿Cuál es la potencia total emitida por un ser humano en $\text{W}$?

        \item La luz solar que incide sobre un cuerpo en la Tierra proporciona $1357\text{ W/m}^2$. 
        ¿Cuál sería la temperatura, en $\text{K}$ y $°\text{C}$, del cuerpo si absorbiera 
        completamente todo este flujo de energía solar?

        \item Un flujo de lava de $2000\text{ K}$ tiene $10 \text{ m}$ de ancho y $100\text{ m}$ de largo. 
        ¿Cuál es la potencia térmica total de esta roca caliente en $\text{MW}$

        \item Una pieza de aluminio de $2 \text{ m}^2$ está pintada de modo que absorbe
        solo el $10\%$ de la energía solar que incide sobre ella ($\text{Albedo} = 0.9$). 
        Si el panel de aluminio está en el exterior de la Estación Espacial Internacional, 
        y el flujo solar en el espacio es de $1357\text{ W/m}^2$, 
        ¿Cuál será la temperatura de equilibrio, en $\text{K}$, $\text{°C}$ y $\text{°F}$, del panel bajo plena luz solar?
    \end{enumerate}
    Fórmulas de conversión: $\text{°C}=\text{K}-273$ y $\text{°F}=\frac{9}{5}\text{°C}+32$
    
\newpage
    
\section{Estrellas de Neutrones}

Las estrellas de neutrones son todo lo que queda de una estrella masiva que explotó como una supernova. Propuestas por primera vez hace más de 50 años, estos cuerpos densos, con apenas 50 kilómetros de diámetro, contienen tanta masa como todo nuestro Sol, que apenas tiene 1 millón de kilómetros de diámetro.

    \begin{figure}[H]
        \centering
        \includegraphics[width=0.7\textwidth]{NeutronStar.jpg}
        \caption{Ilustración de una estrella de neutrones en rotación}
        \label{fig:2}
    \end{figure}

Los astrónomos han estudiado docenas de estas estrellas muertas para determinar cuáles pueden ser los rangos de masa de las estrellas de neutrones. Este rango de masa es una pista importante para comprender cómo se ve el interior de estos cuerpos.

Al estudiar los rayos X emitidos por las estrellas de neutrones y al encontrar muchas que están en sistemas binarios de estrellas, se han "pesado" varias estrellas de neutrones. Cinco de ellas han sido medidas detalladamente para componer los siguientes rangos de masa, donde la masa se da en múltiplos de masas solares ($\text{M}_{\odot} = 2\times 10^{30}\text{ kg}$):

	\begin{table}[H]
   		\centering
        \begin{tabular}{|c|c|}
        	\hline
        	Fuente de emisión de rayos X & Rango de masa [$\text{M}_{\odot}$] \\
            \hline
            3U0900-40 & $1.2 < M < 2.4$ \\
            \hline    
            Centaurus X-3 & $0.7 < M < 4.3$ \\
            \hline
            SMC X-1 & $0.8 < M < 1.8$ \\
            \hline
            Hercules X-1 & $0.0 < M < 2.3$ \\
            \hline
        \end{tabular}
        \caption{Rangos de masa estimados para cinco estrellas de neutrones detectadas mediante su emisión de rayos X en sistemas binarios}
        \label{tab:1}
    \end{table}

    \begin{enumerate}
    	\item ¿Cuál es el punto de intersección de estos límites para las masas de las estrellas de neutrones?
        \item ¿Cuál es el rango de masa permitido para una estrella de neutrones en kilogramos?
    \end{enumerate}


\newpage

\section{Poder de Resolución del Telescopio}

El tamaño de un espejo de telescopio determina qué tan bien puede resolver detalles en objetos distantes.

    \begin{figure}[H]
        \centering
        \includegraphics[width=0.7\textwidth]{MirrorHubble.jpg}
		\caption{Espejo del telescopio espacial Hubble en 1993}
        \label{fig:3}
    \end{figure}
     
Los astrónomos siempre están construyendo telescopios más grandes para ayudarles a ver el universo lejano con mayor claridad.
	\begin{enumerate}
		\item Esta sencilla función predice la resolución $R(D)$,  en segundos de arco ($\text{''}$), de un espejo de telescopio cuyo diámetro, 
		$D$, se da en centímetros:
		
		$$R(D) = \frac{10.3}{D} \text{ ''}$$
		
		Si el dominio de $R(D)$ abarca desde el tamaño de un ojo humano ($0.5\text{ cm}$) hasta el diámetro del Telescopio Espacial Hubble
		 ($240 \text{ cm}$),¿Cuál es el rango angular de $R(D)$ en segundos de arco?
		\item Complete los números faltantes en la forma tabular de $R(D)$ que se muestra a continuación.
		\begin{table}[H]
		     \centering
		     \begin{tabular}{|c|c|c|c|c|c|c|c|c|c|c|}
		         \hline
		         D & & 1 & & 20 & & 100 & & 200 & \\ 
		         \hline
		         R(D) & 21.0 & & 2.1 & & 0.21 & & 0.069 & & 0.043 \\
		         \hline
		     \end{tabular}
		     \caption{Valores de la función $R(D)$ para diferentes diámetros $D$.}
		     \label{tab:2}
		 \end{table} 
		Utilice una precisión de dos cifras significativas, redondeando cuando sea apropiado.
	\end{enumerate}

    
\cite{algebra2}
\bibliographystyle{plainurl}
\bibliography{bibliografia} % Nombre del .bib
\end{document}