\documentclass[a4paper]{article}

\usepackage[spanish]{babel}
\usepackage{graphicx}
\usepackage{amsmath, amssymb}
\usepackage[margin=2.5cm]{geometry}
\usepackage{fancyhdr}
\usepackage{enumerate}
\usepackage[shortlabels]{enumitem}
\usepackage{parskip}
\usepackage[most]{tcolorbox}
\usepackage[hidelinks]{hyperref}
\usepackage{float}
\usepackage{xcolor}
\usepackage{afterpage}
\usepackage{setspace}

% cabecera
\pagestyle{fancy}
\fancyhead[l]{Semillero de Astronomía}
\fancyhead[c]{Banco de Problemas \#1}
\fancyhead[r]{\today}
\fancyfoot[c]{\thepage}
\renewcommand{\headrulewidth}{0.2pt}

% Comando para espacio vertical
\newcommand{\vsp}{\vspace{0.5cm}}

%% Definicion de comandos para formato de unidades fisicas %%
\newcommand{\m}{\text{m}}
\newcommand{\cm}{\text{cm}}
\newcommand{\mm}{\text{mm}}
\newcommand{\km}{\text{km}}
\newcommand{\s}{\text{s}}
\newcommand{\N}{\text{N}}
\newcommand{\g}{\text{g}}
\newcommand{\kg}{\text{kg}}
\newcommand{\Msolar}{\text{M}_\odot}
\newcommand{\Mterrestre}{\text{M}_\oplus}
\newcommand{\AU}{\text{AU}}
\newcommand{\ly}{\text{ly}}
\newcommand{\nT}{\text{nT}}
\newcommand{\keV}{\text{keV}}
\newcommand{\MeV}{\text{MeV}}
\newcommand{\GeV}{\text{GeV}}
\newcommand{\atoms}{\text{atoms}}
\newcommand{\K}{\text{K}}
\newcommand{\J}{\text{J}}
\newcommand{\particles}{\text{particles}}
\newcommand{\protones}{\text{p}^+}
\newcommand{\electrones}{\text{e}^-}
\newcommand{\Gy}{\text{Gy}}
\newcommand{\My}{\text{My}}
\newcommand{\nm}{\text{nm}}
\newcommand{\blank}{\rule{2cm}{0.4pt}\xspace}
\newcommand{\mol}{\text{mol}}
\newcommand{\W}{\text{W}}



\begin{document}

% Portada
\begin{titlepage}
    \centering
    
    \vspace*{2cm}
    
    % Título principal - color negro normal
    {\fontsize{36}{40}\selectfont\bfseries BANCO DE PROBLEMAS}
    
    \vsp
    \vsp
    
    % Subtítulo
    {\Large\textbf{Astronomía para las Ciencias Básicas}}
    
    \vsp
    
    {\large Volumen 1}
    
    \vspace{3cm}
    
    \rule{0.7\textwidth}{0.4pt}
    
    \vspace{2.5cm}
    
    {\LARGE\bfseries Institución Educativa}
    
    \vsp
    
    {\LARGE\bfseries Enrique Olaya Herrera}
    
    \vsp
    
    {\LARGE\bfseries (IEEOH)}
    
    \vspace{2.5cm}
    
    {\Large Una iniciativa del:}
    
    \vspace{0.8cm}
    
    \begin{minipage}{0.8\textwidth}
        \centering
        {\fontsize{20}{24}\selectfont\bfseries
        CENTRO DE INTERÉS:}
        
        \vspace{0.4cm}
        
        {\fontsize{24}{28}\selectfont\bfseries
        SEMILLERO DE ASTRONOMÍA}
    \end{minipage}
    
    \vfill
    
    % Espacio para el autor
    \vspace{1.5cm}
    
    \begin{minipage}{\textwidth}
        \centering
        {\large Autor: Daniel Soto} \\
        \vspace{0.3cm}
        {\large \today}
    \end{minipage}
    
    \nopagecolor
    
\end{titlepage}

\section{Campo magnético terrestre}

    \begin{figure}[H]
        \centering
        \includegraphics[width=0.7\textwidth]{EarthMagneticField.png}
        \caption{Campo magnético de la tierra}
        \label{EarthMagneticField}
    \end{figure}


    \begin{table}[H]
    \centering
    \begin{tabular}{|c|c|c|}
    \hline
    Símbolo & Nombre & Valor \\
    \hline
    $c$ & Velocidad de la Luz & $2.9979\times 10^{10} \text{ cm/s}$\\
    \hline
    $h$ & Constante de Planck & $6.6262\times 10^{-27} \text{ erg}\cdot \text{s}$\\
    \hline
    $m$ & Masa del Electrón & $9.1095\times 10^{-28} \text{ g}$\\
    \hline
    $e$ & Carga del Electrón & $4.80325 \times 10^{-10} \text{esu}$\\
    \hline
    $G$ & Constante de Gravitación & $6.6732 \times 10^{-8} \text{dyn} \cdot \text{cm}^2 \text{gm}^{-2}$\\
    \hline
    $M$ & Masa del Protón & $1.6726 \times 10^{-24} \text{g}$\\
    \hline
    \end{tabular}
    \caption{Constantes físicas}
    \label{ConstantesFisicas}
    \end{table}


\cite{algebra2}
\bibliographystyle{plainurl}
\bibliography{bibliografia} % Nombre del .bib
\end{document}