\documentclass[a4paper]{article} %Formato de plantilla

\usepackage[utf8]{inputenc}
\usepackage[spanish]{babel}
\usepackage[margin=2cm, top=2cm, includefoot]{geometry}
\usepackage{graphicx}
\usepackage[table, xcdraw]{xcolor}
\usepackage{fancyhdr} % Definir estilo de la página
\usepackage[hidelinks]{hyperref} %Gestión de enlaces
\usepackage{parskip}
\usepackage[figurename=FigurA]{caption}
\usepackage{amsmath}
\usepackage{amssymb}

% Adicionales
\addto\captionsspanish{\renewcommand{\contentsname}{Índice}} %Modificacion del indice
\setlength{\headheight}{40.2pt}

%Inicio del documento
\begin{document}
    \cfoot{\thepage}
%----------------------------------------------------------------
% Portada del documento
\begin{titlepage}
    \centering
    % Título principal
    {\scshape\LARGE \textbf{PROPUESTA INSTITUCIONAL}\par}
    \vspace{1cm}
    {\Huge\bfseries SEMILLERO DE ASTRONOMÍA\par}
    \vspace{0.5cm}
    {\LARGE \textbf{Institución Educativa Enrique Olaya Herrera}\par}
    
    \vspace{2cm}
    
    % Información institucional
    {\large \textbf{Presentado a:}\par}
    {\large Rectoría y Consejo Académico\par}
    
    \vspace{1.5cm}
    
    {\large \textbf{Elaborado por:}\par}
    {\large Gerardo Ríos \& Daniel Soto\par}
    
    \vspace{1.5cm}
    
    {\large \textbf{Fecha de presentación:}\par}
    {\large \today \par}
    
    \vfill
    
     {\small \textbf{Versión preliminar para revisión y aprobación}\par}

\end{titlepage}

%----------------------------------------------------------------
% Tabla de contenidos
\clearpage
\tableofcontents
\clearpage

\section{Introducción y Justificación}

Esta propuesta busca establecer el Semillero de Astronomía como una actividad extracurricular formal de la IEEOH. 
La astronomía es una ciencia integradora que fomenta el pensamiento crítico, la aplicación del método científico y una visión interdisciplinaria del conocimiento, 
conectando física, matemática, tecnología y hasta filosofía. Su creación se justifica en la necesidad de:

\begin{itemize}
    \item Ofrecer un espacio de profundización científica más allá del currículo estándar.
    \item Proporcionar una alternativa de aprovechamiento del tiempo libre con alto valor educativo.
    \item Potenciar las competencias científicas e investigativas de los estudiantes, reflejándose positivamente en su desempeño académico general.
\end{itemize}

\section{Objetivos}

\subsection{Objetivo General}
Fomentar la pasión por la ciencia y el pensamiento crítico en los estudiantes a través de la observación, la experimentación y el estudio de la astronomía mediante herramientas matemáticas, formando una comunidad escolar interesada en la investigación, exploración y comprensión del universo.

\subsection{Objetivos Epecíficos}

\begin{itemize}
	\item Capacitar a los estudiantes en conceptos básicos y avanzados de astronomía y astrofísica.
	\item Desarrollar habilidades prácticas en el uso de instrumentos de observación (binoculares, telescopios, software especializado).
	\item Preparar y participar en eventos académicos como la Olimpiada Colombiana de Astronomía y ferias científicas departamentales/nacionales.
	\item Establecer alianzas con instituciones como la Red de Astronomía de Colombia o la Universidad de Antioquia.
\end{itemize}

\section{Metodología y Plan de Trabajo}

\subsection{Modalidad de Trabajo}
El semillero funcionará como un \textit{club extracurricular}, con reuniones semanales de carácter teórico-práctico. 
Se priorizará el aprendizaje colaborativo y por proyectos.

\subsection{Cronograma de Temáticas}
Se define a medida que se avance en las clases

\subsection{Frecuencia y Logística}

\begin{itemize}
	\item Reuniones: Un día a la semana, una o dos horas (Día y hora se definen después).
	\item Lugar: Pendiente a definir dependiendo de la disponibilidad de espacios en la institución
	\item Población Objetivo: Estudiantes de grados 6° a 11° (Puede Modificarse).
\end{itemize}
\section{Organización y Responsabilidades}

\begin{itemize}
    \item Asesor Principal: \textit{Gerardo Ríos}. Responsable de la planeacióny enlace con la institución.
    \item Co-asesor: \textit{Daniel Soto}: Apoyo en logística, dirección académica y temáticas específicas.
    \item Estudiante Líder: Un miembro del semillero elegido por votación para apoyar en organización y comunicación interna. (Pendiente de definir si se crea este rol o no)
    \item Comité de Logística: Grupo rotativo encargado del cuidado de materiales, registro fotográfico y divulgación.
\end{itemize}

\section{Recursos y Necesidades}

\subsection{Recursos Existentes}

\begin{itemize}
    \item Aula/Laboratorio con acceso a internet .
    \item Proyector y computador.
    \item Biblioteca escolar con sección de ciencias.
\end{itemize}

\subsection{Recursos por Gestionar (Priorizados)}

\begin{itemize}
    \item Material Didáctico: Telescopio reflector básico, binoculares, planisferios, cartas celestes.
   \item Bibliografía Especializada: adquisición de libros de astronomía para jóvenes.
    \item Recursos para Salidas
\end{itemize}

\subsection{Alianzas Estratégicas a Gestionar}

\begin{itemize}
    \item Red de Astronomía de Colombia (RAC).
    \item Instituto de Física, Facultad de Ciencias Exactas y Naturales, Universidad de Antioquia
\end{itemize}

\section{Evaluación y Seguimiento}
\begin{itemize}
    \item Asistencia y Participación: Registro en bitácora del semillero.
    \item Productos Tangibles: Evaluación de los proyectos e informes presentados por los estudiantes.
    \item Impacto: Medición a través de la participación en eventos externos (olimpiadas, ferias) y encuestas de satisfacción semestrales a miembros y padres de familia.
    \item Informe Anual: Documento final que presente logros, dificultades y recomendaciones para el siguiente año, presentado a Rectoría.
\end{itemize}

\vspace*{3cm}
    
\textbf{\LARGE PARA APROBACIÓN}
\vspace{2.5cm}
    
    % Firma del/los elaborador(es)
\rule{8cm}{0.4pt}\\
\textbf{Gerardo Ríos \& Daniel Soto}\\
Autores de la propuesta




\end{document}